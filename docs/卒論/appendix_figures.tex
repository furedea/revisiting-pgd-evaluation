\documentclass[a4paper,11pt]{article}
\usepackage[dvipdfmx]{graphicx}
\usepackage[margin=2cm]{geometry}
\usepackage{float}

\title{論文未掲載図(nat\_and\_adv, weak\_advモデルの損失曲線)}
\author{}
\date{}

\begin{document}
\maketitle

本資料は,卒業論文に掲載しなかったnat\_and\_advおよびweak\_advモデルの損失曲線図をまとめたものである.

%%%%%%%%%%%%%%%%%%%%%%%%%%%%%%%%%%%%%%%%%%%%%%%%%%%%%%%%%%%%%%%%%%%%%%%%%%%%%%%
\section{ランダム初期化}

\subsection{MNIST}

\begin{figure}[H]
  \centering
  \includegraphics[width=\linewidth]{figures_ex5/figures/run_all/mnist_nat_and_adv_random_indices0-1-2-3-4_k100_eps0.3_a0.01_r20_seed0.png}
  \caption{MNIST nat\_and\_advモデルに対するランダム初期化PGDの損失曲線.5つのテストサンプルに対し,各20回のリスタートを実行.上段:損失曲線,中段:正誤ヒートマップ,下段:元画像と敵対的サンプル.}
\end{figure}

\begin{figure}[H]
  \centering
  \includegraphics[width=\linewidth]{figures_ex5/figures/run_all/mnist_weak_adv_random_indices0-1-2-3-4_k100_eps0.3_a0.01_r20_seed0.png}
  \caption{MNIST weak\_advモデルに対するランダム初期化PGDの損失曲線.5つのテストサンプルに対し,各20回のリスタートを実行.上段:損失曲線,中段:正誤ヒートマップ,下段:元画像と敵対的サンプル.}
\end{figure}

\subsection{CIFAR10}

\begin{figure}[H]
  \centering
  \includegraphics[width=\linewidth]{figures_ex5/figures/run_all/cifar10_nat_and_adv_random_indices0-1-2-3-4_k100_eps0.03137254901960784_a0.00784313725490196_r20_seed0.png}
  \caption{CIFAR10 nat\_and\_advモデルに対するランダム初期化PGDの損失曲線.5つのテストサンプルに対し,各20回のリスタートを実行.上段:損失曲線,中段:正誤ヒートマップ,下段:元画像と敵対的サンプル.}
\end{figure}

\begin{figure}[H]
  \centering
  \includegraphics[width=\linewidth]{figures_ex5/figures/run_all/cifar10_weak_adv_random_indices0-1-2-3-4_k100_eps0.03137254901960784_a0.00784313725490196_r20_seed0.png}
  \caption{CIFAR10 weak\_advモデルに対するランダム初期化PGDの損失曲線.5つのテストサンプルに対し,各20回のリスタートを実行.上段:損失曲線,中段:正誤ヒートマップ,下段:元画像と敵対的サンプル.}
\end{figure}

%%%%%%%%%%%%%%%%%%%%%%%%%%%%%%%%%%%%%%%%%%%%%%%%%%%%%%%%%%%%%%%%%%%%%%%%%%%%%%%
\section{DeepFool初期化}

\subsection{MNIST}

\begin{figure}[H]
  \centering
  \includegraphics[width=\linewidth]{figures_ex5/figures/run_all/mnist_nat_and_adv_deepfool_indices0-1-2-3-4_k100_eps0.3_a0.01_r1_seed0_dfiter50_dfo0.02_dfj0.0_dfproject_clip.png}
  \caption{MNIST nat\_and\_advモデルに対するDeepFool初期化PGDの損失曲線.5つのテストサンプルに対し,DeepFool初期点からPGD攻撃を実行.上段:損失曲線,下段:元画像,DeepFool結果,最終敵対的サンプル.}
\end{figure}

\begin{figure}[H]
  \centering
  \includegraphics[width=\linewidth]{figures_ex5/figures/run_all/mnist_weak_adv_deepfool_indices0-1-2-3-4_k100_eps0.3_a0.01_r1_seed0_dfiter50_dfo0.02_dfj0.0_dfproject_clip.png}
  \caption{MNIST weak\_advモデルに対するDeepFool初期化PGDの損失曲線.5つのテストサンプルに対し,DeepFool初期点からPGD攻撃を実行.上段:損失曲線,下段:元画像,DeepFool結果,最終敵対的サンプル.}
\end{figure}

\subsection{CIFAR10}

\begin{figure}[H]
  \centering
  \includegraphics[width=\linewidth]{figures_ex5/figures/run_all/cifar10_nat_and_adv_deepfool_indices0-1-2-3-4_k100_eps0.03137254901960784_a0.00784313725490196_r1_seed0_dfiter50_dfo0.02_dfj0.0_dfproject_clip.png}
  \caption{CIFAR10 nat\_and\_advモデルに対するDeepFool初期化PGDの損失曲線.5つのテストサンプルに対し,DeepFool初期点からPGD攻撃を実行.上段:損失曲線,下段:元画像,DeepFool結果,最終敵対的サンプル.}
\end{figure}

\begin{figure}[H]
  \centering
  \includegraphics[width=\linewidth]{figures_ex5/figures/run_all/cifar10_weak_adv_deepfool_indices0-1-2-3-4_k100_eps0.03137254901960784_a0.00784313725490196_r1_seed0_dfiter50_dfo0.02_dfj0.0_dfproject_clip.png}
  \caption{CIFAR10 weak\_advモデルに対するDeepFool初期化PGDの損失曲線.5つのテストサンプルに対し,DeepFool初期点からPGD攻撃を実行.上段:損失曲線,下段:元画像,DeepFool結果,最終敵対的サンプル.}
\end{figure}

%%%%%%%%%%%%%%%%%%%%%%%%%%%%%%%%%%%%%%%%%%%%%%%%%%%%%%%%%%%%%%%%%%%%%%%%%%%%%%%
\section{Multi-DeepFool初期化}

\subsection{MNIST}

\begin{figure}[H]
  \centering
  \includegraphics[width=\linewidth]{figures_ex5/figures/run_all/mnist_nat_and_adv_multi_deepfool_indices0-1-2-3-4_k100_eps0.3_a0.01_r9_seed0_dfiter50_dfo0.02.png}
  \caption{MNIST nat\_and\_advモデルに対するMulti-DeepFool初期化PGDの損失曲線.5つのテストサンプルに対し,9つのターゲットクラスへのDeepFool初期点からPGD攻撃を実行.上段:損失曲線(9本),中段:正誤ヒートマップ,下段:元画像,最近傍ターゲットへのDeepFool結果,最終敵対的サンプル.}
\end{figure}

\begin{figure}[H]
  \centering
  \includegraphics[width=\linewidth]{figures_ex5/figures/run_all/mnist_weak_adv_multi_deepfool_indices0-1-2-3-4_k100_eps0.3_a0.01_r9_seed0_dfiter50_dfo0.02.png}
  \caption{MNIST weak\_advモデルに対するMulti-DeepFool初期化PGDの損失曲線.5つのテストサンプルに対し,9つのターゲットクラスへのDeepFool初期点からPGD攻撃を実行.上段:損失曲線(9本),中段:正誤ヒートマップ,下段:元画像,最近傍ターゲットへのDeepFool結果,最終敵対的サンプル.}
\end{figure}

\subsection{CIFAR10}

\begin{figure}[H]
  \centering
  \includegraphics[width=\linewidth]{figures_ex5/figures/run_all/cifar10_nat_and_adv_multi_deepfool_indices0-1-2-3-4_k100_eps0.03137254901960784_a0.00784313725490196_r9_seed0_dfiter50_dfo0.02.png}
  \caption{CIFAR10 nat\_and\_advモデルに対するMulti-DeepFool初期化PGDの損失曲線.5つのテストサンプルに対し,9つのターゲットクラスへのDeepFool初期点からPGD攻撃を実行.上段:損失曲線(9本),中段:正誤ヒートマップ,下段:元画像,最近傍ターゲットへのDeepFool結果,最終敵対的サンプル.}
\end{figure}

\begin{figure}[H]
  \centering
  \includegraphics[width=\linewidth]{figures_ex5/figures/run_all/cifar10_weak_adv_multi_deepfool_indices0-1-2-3-4_k100_eps0.03137254901960784_a0.00784313725490196_r9_seed0_dfiter50_dfo0.02.png}
  \caption{CIFAR10 weak\_advモデルに対するMulti-DeepFool初期化PGDの損失曲線.5つのテストサンプルに対し,9つのターゲットクラスへのDeepFool初期点からPGD攻撃を実行.上段:損失曲線(9本),中段:正誤ヒートマップ,下段:元画像,最近傍ターゲットへのDeepFool結果,最終敵対的サンプル.}
\end{figure}

\end{document}
