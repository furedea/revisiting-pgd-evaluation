\chapter{評価実験}
本章では,Madryらの実験の再現及び提案手法であるDeepFool初期化の評価実験について述べる.

\section{再現実験}
\label{sec:reproduction}

\subsection{実験設定}

\subsubsection{使用モデル}
Madryらが公開しているリポジトリ\cite{MadryMNIST, MadryCIFAR10}から入手した事前学習済みモデルを使用する.具体的には以下の4つのモデルを用いる:
\begin{itemize}
    \item MNISTの自然学習モデル(Naturally trained)
    \item MNISTの敵対的学習モデル(Adversarially trained)
    \item CIFAR10の自然学習モデル(Naturally trained)
    \item CIFAR10の敵対的学習モデル(Adversarially trained)
\end{itemize}

\subsubsection{攻撃パラメータ}
表\ref{table:attack_params}にPGD攻撃のパラメータを示す.

\begin{table}[hbtp]
  \caption{PGD攻撃のパラメータ設定}
  \label{table:attack_params}
  \centering
  \begin{tabular}{l|cc}
    \hline
    パラメータ & MNIST & CIFAR10 \\
    \hline
    摂動制約$\varepsilon$ & $0.3$ & $8/255$ \\
    ステップサイズ$\alpha$ & $0.01$ & $2/255$ \\
    反復数$T$ & $100$ & $100$ \\
    リスタート数 & $20$ & $20$ \\
    \hline
  \end{tabular}
\end{table}

\subsubsection{実行環境}
% TODO: 実行環境の詳細を記載

\subsection{実験結果}

% TODO: 再現実験の結果を追記
% - MNISTの自然学習モデルでの結果
% - MNISTの敵対的学習モデルでの結果
% - CIFAR10の自然学習モデルでの結果
% - CIFAR10の敵対的学習モデルでの結果
% - 論文との比較・考察

\section{DeepFool初期化実験}
\label{sec:deepfool_experiment}

\subsection{実験設定}

\subsubsection{DeepFoolパラメータ}
DeepFoolのパラメータは以下のように設定する:
\begin{itemize}
    \item 最大反復回数$T_{\text{df}} = 50$
    \item オーバーシュート係数$\eta = 0.02$
    \item ジッターjitter $= 0.0$(デフォルト)
\end{itemize}

\subsection{実験結果}

% TODO: DeepFool初期化実験の結果を追記
% - 各モデルでのDeepFool初期化結果
% - クリッピング vs スケーリングの比較
% - Multi-DeepFool初期化(Maxloss選択)の結果
% - ランダム初期化との比較

\section{考察}
\label{sec:discussion}

% TODO: 実験結果の考察を追記
% - DeepFool初期化の問題点
% - 損失関数との関係
% - 射影方法の影響

\section{本実験の方針(今後)}
\label{sec:future_experiments}

本研究では,初期化手法と反復回数の関係をより系統的に調査するため,以下の実験を計画している.

\subsection{比較対象}
\begin{itemize}
    \item 入力点初期化(Input initialization)
    \item ランダム初期化(Random initialization)
    \item DeepFool初期化(DeepFool initialization)
\end{itemize}

\subsection{評価指標}
\begin{itemize}
    \item 反復毎の損失曲線
    \item プラトー到達反復数の分布
    \item Robust accuracy(敵対的サンプルに対する正答率)
\end{itemize}
