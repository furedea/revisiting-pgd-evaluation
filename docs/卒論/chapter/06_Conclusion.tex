\chapter{結論}

\section{本研究の成果}

本研究では,PGD攻撃によるロバスト性評価の信頼性について,反復回数と初期化手法の観点から再検証を行った.

\subsection{PGD攻撃の収束性検証}
Madryらが報告したPGD攻撃の損失収束に関する実験の再現を行い,定量的な評価指標を用いて拡張した.収束閾値$\theta = 0.90$を用いた評価により,PGD攻撃が安定して局所最適解に到達することを確認した.

主な知見として以下が得られた:
\begin{itemize}
    \item 異なるランダム初期点から開始しても,損失曲線は同程度のプラトーに収束する
    \item 自然学習モデルは敵対的学習モデルと比較して収束が早い傾向がある
    \item MNISTとCIFAR10で同様の傾向が確認された
\end{itemize}

% TODO: 実験結果に基づいて具体的な数値を追記

\subsection{DeepFool初期化手法の提案と評価}
PGD攻撃の新たな初期化手法として,DeepFoolの境界情報を用いた初期化を提案し,その効果を検証した.

主な知見として以下が得られた:
\begin{itemize}
    \item 自然学習モデルでは,DeepFool初期化により収束の高速化が観察された
    \item 敵対的学習モデルでは,効果が限定的であった
    \item Multi-DeepFool初期化は,単一ターゲットのDeepFool初期化と比較して改善が見られた
\end{itemize}

% TODO: 実験結果に基づいて具体的な数値を追記

\subsection{実験的知見}
本研究を通じて,以下の実験的知見が得られた:
\begin{itemize}
    \item 90\%収束閾値を用いた場合,ランダム初期化PGDでは平均35反復程度で収束に到達する
    \item P95(95パーセンタイル)では約90反復が必要であり,ほぼ全てのサンプルの収束には十分な反復回数が必要である
    \item モデルのロバスト性(敵対的学習の有無)は収束速度に影響を与える
\end{itemize}

% TODO: 実験結果に基づいて具体的な数値を修正

\section{今後の展望}

本研究を発展させる方向として,以下の課題が考えられる.

\begin{itemize}
    \item \textbf{他のデータセット・モデルへの適用}:ImageNetなどのより大規模なデータセットや,異なるアーキテクチャのモデルへの適用可能性を検証する
    \item \textbf{他の攻撃手法との比較}:AutoAttackやCW攻撃など,他の強力な攻撃手法との収束性比較を行う
    \item \textbf{適応的な反復回数の決定}:収束性の分析結果に基づいて,サンプルやモデルに応じた適応的な反復回数決定手法の開発
    \item \textbf{計算効率の改善}:DeepFool初期化の計算コストを削減しつつ,収束高速化の効果を維持する手法の検討
\end{itemize}

