\chapter{評価方法}
\label{chapter:evaluation_method}
本章では,PGD攻撃の誤分類性能を定量的に評価するための方法について述べる.

\section{誤分類の判定}
\label{sec:misclassification_definition}

PGD攻撃の目的は,モデルに誤分類を引き起こすことである.本節では,攻撃成功(誤分類)の判定基準と,評価に用いる「誤分類達成反復数」の定義を与える.

\subsection{攻撃成功と攻撃失敗}
\label{sec:attack_success_failure}

本研究では,最大反復数$T = 100$とし,最終反復時点での分類結果に基づいて攻撃の成否を判定する:
\begin{itemize}
    \item \textbf{攻撃成功}:最終反復時点で誤分類している場合
    \item \textbf{攻撃失敗}:最終反復時点で正しく分類されている場合
\end{itemize}

攻撃失敗は,モデルが敵対的摂動に対して堅牢であることを示唆する.特に敵対的訓練を施したモデルでは,PGD攻撃によっても誤分類が困難な場合がある.

\subsection{誤分類達成反復数の定義}
\label{sec:first_misclassification_iteration}

PGD攻撃の効果を定量的に評価するため,\textbf{誤分類達成反復数}を導入する.これは,モデルの予測が正解ラベルから不正解ラベルに変わった最初の反復数である.

$t$回目の反復における敵対的サンプルを$x^t$,正解ラベルを$y$,敵対的サンプルに対するモデルの予測ラベルを$k(x)$とすると,誤分類達成反復数$t^*$は式(\ref{eq:first_misclassification})で定義する.
\begin{align}
    t^* = \min\{t : k(x^t) \neq y\}
    \label{eq:first_misclassification}
\end{align}

この定義により,攻撃がいつ成功したのかを直接的に測定できる.損失値に基づく間接的な指標ではなく,分類結果という最終的な目標に基づいた評価が可能となる.

\section{評価指標}
\label{sec:evaluation_metrics}

誤分類の定量的評価のため,以下の指標を用いる.

\subsection{誤分類達成率}
\label{sec:misclassification_achievement_rate}

誤分類達成率は,PGD攻撃の試行のうち,最終反復時点で誤分類している試行の割合を表す.ここで,1つの試行は1つの入力サンプルと1つの初期点の組み合わせに対する1回のPGD攻撃を指す.例えば,100サンプルに対してそれぞれ20回のランダム初期化を行う場合,総試行数は$100 \times 20 = 2000$となる.

総試行数を$N$,最終反復時点で誤分類している試行数を$N_m$とすると,誤分類達成率$R_m$は式(\ref{eq:misclassification_achievement_rate})で定義する.

\begin{align}
    R_m = \frac{N_m}{N}
    \label{eq:misclassification_achievement_rate}
\end{align}

本研究では最大反復数$T = 100$を設定している.$R_m = 1.0$であれば全試行が最終反復時点で誤分類していることを意味し,$R_m < 1.0$の場合は一部の試行で攻撃が失敗したことを示す.

\paragraph{攻撃成功率との違い}
敵対的学習の文脈では,「攻撃成功率」は一般に「$N$個のサンプルのうち,少なくとも1回の試行で誤分類を達成したサンプルの割合」として定義されることが多い.これはサンプル単位での攻撃の有効性を評価する指標である.

一方,本研究で用いる「誤分類達成率」は,サンプルと初期点の全組み合わせに対する試行単位での成功割合を表す.この指標は,初期化手法の信頼性や安定性を比較する上で有用である.例えば,2000試行のうち100試行で誤分類を達成した場合,誤分類達成率は5\%となり,攻撃の困難さを詳細に反映できる.

\subsection{誤分類達成反復数の統計量}
\label{sec:misclassification_statistics}

誤分類達成率は全体的な成功率を示すが,誤分類に要した反復数の分布については情報を与えない.そこで,誤分類達成反復数の分布を特徴づけるため,以下の統計量を計算する:
\begin{itemize}
    \item \textbf{平均}:誤分類達成反復数の算術平均.全体の目安となる.
    \item \textbf{中央値(Median)}:誤分類達成反復数の中央値.外れ値の影響を受けにくい.
    \item \textbf{95パーセンタイル(P95)}:誤分類達成反復数の分布において95\%の試行が誤分類に到達する反復数.最悪ケースに近い.
\end{itemize}

これらの統計量を組み合わせることで,誤分類達成反復数の分布を多面的に把握できる.例えば,平均と中央値が大きく異なる場合は,一部の試行で誤分類が著しく遅いことを示唆する.P95は,ほぼ全ての試行が誤分類に到達するために必要な反復数の実用的な目安として重要である.
