\chapter{評価方法}
\label{chapter:evaluation_method}
本章では,PGD攻撃の誤分類性能を定量的に評価するための方法について述べる.

\section{誤分類の判定}
\label{sec:misclassification_definition}

PGD攻撃の目的は,モデルに誤分類を引き起こすことである.本節では,攻撃成功(誤分類)の判定基準と,評価に用いる「初回誤分類反復数」の定義を与える.

\subsection{初回誤分類反復数の定義}
\label{sec:first_misclassification_iteration}

PGD攻撃の効果を定量的に評価するため,\textbf{初回誤分類反復数}を導入する.これは,モデルの予測が正解ラベルから不正解ラベルに変わった最初の反復数である.

$t$回目の反復における敵対的サンプルを$x^t$,正解ラベルを$y$,敵対的サンプルに対するモデルの予測ラベルを$k(x)$とすると,初回誤分類反復数$t^*$は式(\ref{eq:first_misclassification})で定義する.
\begin{align}
    t^* = \min\{t : k(x^t) \neq y\}
    \label{eq:first_misclassification}
\end{align}

この定義により,攻撃が「いつ成功したか」を直接的に測定できる.損失値に基づく間接的な指標ではなく,分類結果という最終的な目標に基づいた評価が可能となる.

\subsection{攻撃成功と攻撃失敗}
\label{sec:attack_success_failure}

本研究では,最大反復数$T = 100$とし,最終反復時点での分類結果に基づいて攻撃の成否を判定する:
\begin{itemize}
    \item \textbf{攻撃成功}:最終反復時点で誤分類している場合
    \item \textbf{攻撃失敗}:最終反復時点で正しく分類されている場合
\end{itemize}

攻撃失敗は,モデルが敵対的摂動に対して堅牢であることを示唆する.特に敵対的訓練を施したモデルでは,PGD攻撃によっても誤分類が困難な場合がある.

\section{評価指標}
\label{sec:evaluation_metrics}

誤分類の定量的評価のため,以下の指標を用いる.

\subsection{攻撃成功率}
\label{sec:attack_success_rate}

攻撃成功率は,PGD攻撃の試行のうち,最終反復時点で誤分類している試行の割合を表す.ここで,1つの試行は1つの異なる初期点から開始する1回のPGD攻撃を指す.例えばランダム初期化では,異なるランダム初期点から複数回の攻撃を行い,各攻撃を独立な試行として数える.

総試行数を$N$,最終反復時点で誤分類している試行数を$N_a$とすると,攻撃成功率$R_a$は式(\ref{eq:attack_success_rate})で定義する.

\begin{align}
    R_a = \frac{N_a}{N}
    \label{eq:attack_success_rate}
\end{align}

本研究では最大反復数$T = 100$を設定している.$R_a = 1.0$であれば全試行が最終反復時点で誤分類していることを意味し,$R_a < 1.0$の場合は一部の試行で攻撃が失敗したことを示す.

\subsection{初回誤分類反復数の統計量}
\label{sec:misclassification_statistics}

攻撃成功率は全体的な成功率を示すが,誤分類に要した反復数の分布については情報を与えない.そこで,初回誤分類反復数の分布を特徴づけるため,以下の統計量を計算する:
\begin{itemize}
    \item \textbf{平均}:初回誤分類反復数の算術平均.全体の目安となる.
    \item \textbf{中央値(Median)}:初回誤分類反復数の中央値.外れ値の影響を受けにくい.
    \item \textbf{95パーセンタイル(P95)}:初回誤分類反復数の分布において95\%の試行が誤分類に到達する反復数.最悪ケースに近い.
\end{itemize}

これらの統計量を組み合わせることで.初回誤分類反復数の分布を多面的に把握できる.例えば,平均と中央値が大きく異なる場合は,一部の試行で誤分類が著しく遅いことを示唆する.P95は,ほぼ全ての試行が誤分類に到達するために必要な反復数の実用的な目安として重要である.

\subsection{累積分布関数(CDF)による可視化}
\label{sec:cdf}

初回誤分類反復数の分布を可視化するため,累積分布関数(Cumulative Distribution Function; CDF)を用いる.異なる初期点で複数の試行を行った場合,各試行で初回誤分類反復数が異なりうる.誤分類CDFは,ある反復数$t$までに誤分類を達成した試行の割合を表し,試行間での初回誤分類反復数のばらつきを可視化できる.

CDFが急激に立ち上がる場合は多くの試行が短い反復数で誤分類に到達していることを示し,緩やかに立ち上がる場合は誤分類に要する反復数のばらつきが大きいことを示す.
