\chapter{nat\_and\_advおよびweak\_advモデルの損失曲線}
\label{appendix:loss_curves}

本付録では,本文で省略したnat\_and\_advモデルおよびweak\_advモデルに対するPGD攻撃の損失曲線を示す.図の構成は本文(第\ref{sec:loss_curve_observation}節)と同様である.

%%%%%%%%%%%%%%%%%%%%%%%%%%%%%%%%%%%%%%%%%%%%%%%%%%%%%%%%%%%%%%%%%%%%%%%%%%%%%%%
\section{ランダム初期化}

\subsection{MNIST nat\_and\_advモデル}

図\ref{fig:mnist_nat_and_adv_random}にMNIST nat\_and\_advモデルに対するランダム初期化PGDの結果を示す.

\begin{figure}[H]
  \centering
  \includegraphics[width=\linewidth]{figure/mnist_nat_and_adv_random.png}
  \caption{MNIST nat\_and\_advモデルに対するランダム初期化PGD攻撃の結果.上段:損失曲線(横軸は反復数,縦軸は損失値,各線は異なる試行).中段:正誤ヒートマップ(黄色は正しい分類,紫色は誤分類).下段:元画像,ランダム初期化サンプル,PGD敵対的サンプルの比較.}
  \label{fig:mnist_nat_and_adv_random}
\end{figure}

\medskip
図\ref{fig:mnist_nat_and_adv_random}より,損失曲線を見ると,全サンプルで損失値は$10^{-4}$〜$10^{-3}$程度の非常に小さい範囲に留まっている.ヒートマップを見ると,全5サンプルにおいて20回の試行全てが100反復を通じて正分類(黄色)を維持しており,誤分類を達成した試行は存在しない.

\subsection{MNIST weak\_advモデル}

図\ref{fig:mnist_weak_adv_random}にMNIST weak\_advモデルに対するランダム初期化PGDの結果を示す.

\begin{figure}[H]
  \centering
  \includegraphics[width=\linewidth]{figure/mnist_weak_adv_random.png}
  \caption{MNIST weak\_advモデルに対するランダム初期化PGD攻撃の結果.上段:損失曲線(横軸は反復数,縦軸は損失値,各線は異なる試行).中段:正誤ヒートマップ(黄色は正しい分類,紫色は誤分類).下段:元画像,ランダム初期化サンプル,PGD敵対的サンプルの比較.}
  \label{fig:mnist_weak_adv_random}
\end{figure}

\medskip
図\ref{fig:mnist_weak_adv_random}より,損失曲線を見ると,全サンプルで損失は0付近から開始し,10〜30程度まで上昇している.ヒートマップを見ると,サンプル1, 3, 4, 5では全試行が反復20〜35程度で誤分類(紫色)を達成している.サンプル2では一部の試行(約5〜6試行)が100反復でも正分類を維持している.

\subsection{CIFAR10 nat\_and\_advモデル}

図\ref{fig:cifar10_nat_and_adv_random}にCIFAR10 nat\_and\_advモデルに対するランダム初期化PGDの結果を示す.

\begin{figure}[H]
  \centering
  \includegraphics[width=\linewidth]{figure/cifar10_nat_and_adv_random.png}
  \caption{CIFAR10 nat\_and\_advモデルに対するランダム初期化PGD攻撃の結果.上段:損失曲線(横軸は反復数,縦軸は損失値,各線は異なる試行).中段:正誤ヒートマップ(黄色は正しい分類,紫色は誤分類).下段:元画像,ランダム初期化サンプル,PGD敵対的サンプルの比較.}
  \label{fig:cifar10_nat_and_adv_random}
\end{figure}

\medskip
図\ref{fig:cifar10_nat_and_adv_random}より,損失曲線を見ると,サンプル1, 2, 4, 5では損失が0付近から1〜6程度まで急速に上昇している.サンプル3では損失が0.05程度と非常に低い範囲に留まっている.ヒートマップを見ると,サンプル1, 2, 4, 5では全試行が反復1〜15ほどで誤分類を達成している.サンプル3では全試行が100反復を通じて正分類を維持している.

\subsection{CIFAR10 weak\_advモデル}

図\ref{fig:cifar10_weak_adv_random}にCIFAR10 weak\_advモデルに対するランダム初期化PGDの結果を示す.

\begin{figure}[H]
  \centering
  \includegraphics[width=\linewidth]{figure/cifar10_weak_adv_random.png}
  \caption{CIFAR10 weak\_advモデルに対するランダム初期化PGD攻撃の結果.上段:損失曲線(横軸は反復数,縦軸は損失値,各線は異なる試行).中段:正誤ヒートマップ(黄色は正しい分類,紫色は誤分類).下段:元画像,ランダム初期化サンプル,PGD敵対的サンプルの比較.}
  \label{fig:cifar10_weak_adv_random}
\end{figure}

\medskip
図\ref{fig:cifar10_weak_adv_random}より,損失曲線を見ると,サンプル1, 2, 4, 5では損失が0付近から3.5〜12程度まで急速に上昇している.サンプル3では損失が0.01程度と非常に低い範囲に留まっている.ヒートマップを見ると,サンプル1, 2, 4, 5では全試行が反復1〜10ほどで誤分類を達成している.サンプル3では全試行が100反復を通じて正分類を維持している.

%%%%%%%%%%%%%%%%%%%%%%%%%%%%%%%%%%%%%%%%%%%%%%%%%%%%%%%%%%%%%%%%%%%%%%%%%%%%%%%
\section{DeepFool初期化}

\subsection{MNIST nat\_and\_advモデル}

図\ref{fig:mnist_nat_and_adv_deepfool}にMNIST nat\_and\_advモデルに対するDeepFool初期化PGDの結果を示す.

\begin{figure}[H]
  \centering
  \includegraphics[width=\linewidth]{figure/mnist_nat_and_adv_deepfool.png}
  \caption{MNIST nat\_and\_advモデルに対するDeepFool初期化PGD攻撃の結果.上段:損失曲線(横軸は反復数,縦軸は損失値).中段:正誤ヒートマップ(黄色は正しい分類,紫色は誤分類).下段:元画像,DeepFool敵対的サンプル,PGD敵対的サンプルの比較.}
  \label{fig:mnist_nat_and_adv_deepfool}
\end{figure}

\medskip
図\ref{fig:mnist_nat_and_adv_deepfool}より,損失曲線を見ると,サンプル1, 2, 4では損失が$10^{-4}$〜$10^{-3}$程度の非常に小さい範囲に留まっている.サンプル3, 5では損失が5〜8程度まで上昇している.ヒートマップを見ると,サンプル1, 2, 4は100反復を通じて正分類を維持している.サンプル3, 5は反復10付近で誤分類を達成している.

\subsection{MNIST weak\_advモデル}

図\ref{fig:mnist_weak_adv_deepfool}にMNIST weak\_advモデルに対するDeepFool初期化PGDの結果を示す.

\begin{figure}[H]
  \centering
  \includegraphics[width=\linewidth]{figure/mnist_weak_adv_deepfool.png}
  \caption{MNIST weak\_advモデルに対するDeepFool初期化PGD攻撃の結果.上段:損失曲線(横軸は反復数,縦軸は損失値).中段:正誤ヒートマップ(黄色は正しい分類,紫色は誤分類).下段:元画像,DeepFool敵対的サンプル,PGD敵対的サンプルの比較.}
  \label{fig:mnist_weak_adv_deepfool}
\end{figure}

\medskip
図\ref{fig:mnist_weak_adv_deepfool}より,損失曲線を見ると,全サンプルで損失は0付近から開始し,12〜22程度まで上昇している.ヒートマップを見ると,全サンプルで反復15〜30程度で誤分類を達成している.

\subsection{CIFAR10 nat\_and\_advモデル}

図\ref{fig:cifar10_nat_and_adv_deepfool}にCIFAR10 nat\_and\_advモデルに対するDeepFool初期化PGDの結果を示す.

\begin{figure}[H]
  \centering
  \includegraphics[width=\linewidth]{figure/cifar10_nat_and_adv_deepfool.png}
  \caption{CIFAR10 nat\_and\_advモデルに対するDeepFool初期化PGD攻撃の結果.上段:損失曲線(横軸は反復数,縦軸は損失値).中段:正誤ヒートマップ(黄色は正しい分類,紫色は誤分類).下段:元画像,DeepFool敵対的サンプル,PGD敵対的サンプルの比較.}
  \label{fig:cifar10_nat_and_adv_deepfool}
\end{figure}

\medskip
図\ref{fig:cifar10_nat_and_adv_deepfool}より,損失曲線を見ると,サンプル1, 2, 4, 5では損失が1〜6程度の値に達し,ほぼ横ばいで推移している.サンプル3では損失が0.05程度と非常に低い範囲に留まっている.ヒートマップを見ると,サンプル1, 2, 4, 5は反復1〜2程度で誤分類を達成している.サンプル3は100反復を通じて正分類を維持している.

\subsection{CIFAR10 weak\_advモデル}

図\ref{fig:cifar10_weak_adv_deepfool}にCIFAR10 weak\_advモデルに対するDeepFool初期化PGDの結果を示す.

\begin{figure}[H]
  \centering
  \includegraphics[width=\linewidth]{figure/cifar10_weak_adv_deepfool.png}
  \caption{CIFAR10 weak\_advモデルに対するDeepFool初期化PGD攻撃の結果.上段:損失曲線(横軸は反復数,縦軸は損失値).中段:正誤ヒートマップ(黄色は正しい分類,紫色は誤分類).下段:元画像,DeepFool敵対的サンプル,PGD敵対的サンプルの比較.}
  \label{fig:cifar10_weak_adv_deepfool}
\end{figure}

\medskip
図\ref{fig:cifar10_weak_adv_deepfool}より,損失曲線を見ると,サンプル1, 2, 5では損失が3.5〜12程度の値まで上昇してからはほぼ横ばいで推移している.サンプル3では損失が0.01程度,サンプル4では損失が0.001程度と低い.ヒートマップを見ると,サンプル1, 2, 4, 5は反復1で誤分類を達成している.サンプル3, 4は100反復を通じて正分類を維持している.

%%%%%%%%%%%%%%%%%%%%%%%%%%%%%%%%%%%%%%%%%%%%%%%%%%%%%%%%%%%%%%%%%%%%%%%%%%%%%%%
\section{Multi-DeepFool初期化}

\subsection{MNIST nat\_and\_advモデル}

図\ref{fig:mnist_nat_and_adv_mdf}にMNIST nat\_and\_advモデルに対するMulti-DeepFool初期化PGDの結果を示す.

\begin{figure}[H]
  \centering
  \includegraphics[width=\linewidth]{figure/mnist_nat_and_adv_mdf.png}
  \caption{MNIST nat\_and\_advモデルに対するMulti-DeepFool初期化PGD攻撃の結果.上段:損失曲線(横軸は反復数,縦軸は損失値,各線の色は境界までの距離の順位を表す).中段:正誤ヒートマップ(黄色は正しい分類,紫色は誤分類).下段:元画像,損失が最大となるDeepFool敵対的サンプル,PGD敵対的サンプルの比較.}
  \label{fig:mnist_nat_and_adv_mdf}
\end{figure}

\medskip
図\ref{fig:mnist_nat_and_adv_mdf}より,損失曲線を見ると,サンプル1, 2, 4では損失が$10^{-3}$〜$10^{-2}$程度の非常に小さい範囲に留まっている.サンプル3, 5では損失が2〜8程度まで上昇する試行がある.ヒートマップを見ると,サンプル1, 2, 4は全試行・全反復で正分類を維持している.サンプル3は高ランクになればなるほど誤分類しにくくなり上位の3ラベルについては正分類を維持している.サンプル5は誤分類,正分類が混在しているが,ランクの高低による明確な傾向は見られない.

\subsection{MNIST weak\_advモデル}

図\ref{fig:mnist_weak_adv_mdf}にMNIST weak\_advモデルに対するMulti-DeepFool初期化PGDの結果を示す.

\begin{figure}[H]
  \centering
  \includegraphics[width=\linewidth]{figure/mnist_weak_adv_mdf.png}
  \caption{MNIST weak\_advモデルに対するMulti-DeepFool初期化PGD攻撃の結果.上段:損失曲線(横軸は反復数,縦軸は損失値,各線の色は境界までの距離の順位を表す).中段:正誤ヒートマップ(黄色は正しい分類,紫色は誤分類).下段:元画像,損失が最大となるDeepFool敵対的サンプル,PGD敵対的サンプルの比較.}
  \label{fig:mnist_weak_adv_mdf}
\end{figure}

\medskip
図\ref{fig:mnist_weak_adv_mdf}より,損失曲線を見ると,全サンプルで損失は0付近から開始し,8〜25程度まで上昇している.ヒートマップを見ると,サンプル1, 3, 4, 5では全試行が反復20〜40程度で誤分類を達成している.サンプル2では一部の試行が100反復でも正分類を維持している.ランクの高低による明確な傾向は見られない.

\subsection{CIFAR10 nat\_and\_advモデル}

図\ref{fig:cifar10_nat_and_adv_mdf}にCIFAR10 nat\_and\_advモデルに対するMulti-DeepFool初期化PGDの結果を示す.

\begin{figure}[H]
  \centering
  \includegraphics[width=\linewidth]{figure/cifar10_nat_and_adv_mdf.png}
  \caption{CIFAR10 nat\_and\_advモデルに対するMulti-DeepFool初期化PGD攻撃の結果.上段:損失曲線(横軸は反復数,縦軸は損失値,各線の色は境界までの距離の順位を表す).中段:正誤ヒートマップ(黄色は正しい分類,紫色は誤分類).下段:元画像,損失が最大となるDeepFool敵対的サンプル,PGD敵対的サンプルの比較.}
  \label{fig:cifar10_nat_and_adv_mdf}
\end{figure}

\medskip
図\ref{fig:cifar10_nat_and_adv_mdf}より,損失曲線を見ると,サンプル1, 2, 4, 5では損失が1〜7程度まで上昇しほぼ横ばいで推移している.サンプル3では損失が0.05程度と非常に低い範囲に留まっている.ヒートマップを見ると,サンプル1, 2, 4, 5はほぼ全試行が反復1〜5程度で誤分類を達成しているが,サンプル1の高ランクの試行において1度だけ70反復程度まで誤分類が行われていないケースが存在する.サンプル3は全試行が100反復を通じて正分類を維持している.

\subsection{CIFAR10 weak\_advモデル}

図\ref{fig:cifar10_weak_adv_mdf}にCIFAR10 weak\_advモデルに対するMulti-DeepFool初期化PGDの結果を示す.

\begin{figure}[H]
  \centering
  \includegraphics[width=\linewidth]{figure/cifar10_weak_adv_mdf.png}
  \caption{CIFAR10 weak\_advモデルに対するMulti-DeepFool初期化PGD攻撃の結果.上段:損失曲線(横軸は反復数,縦軸は損失値,各線の色は境界までの距離の順位を表す).中段:正誤ヒートマップ(黄色は正しい分類,紫色は誤分類).下段:元画像,損失が最大となるDeepFool敵対的サンプル,PGD敵対的サンプルの比較.}
  \label{fig:cifar10_weak_adv_mdf}
\end{figure}

\medskip
図\ref{fig:cifar10_weak_adv_mdf}より,損失曲線を見ると,サンプル1, 2, 4, 5では損失が2〜12程度まで上昇し,ほぼ横ばいで推移している.サンプル3では損失が0.01程度と非常に低い範囲に留まっている.ヒートマップを見ると,サンプル1, 2, 5は反復0〜2で誤分類を達成している.サンプル4は反復2〜30程度で誤分類といった,ムラがあるがランクの高低による明確な傾向は見られない.サンプル3は全試行が100反復を通じて正分類を維持している.
