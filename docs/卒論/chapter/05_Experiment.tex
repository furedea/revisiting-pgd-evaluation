\chapter{評価実験}
\label{chapter:experiment}
本章では,Madryらの実験の再現及び提案手法であるDeepFool初期化の評価実験について述べる.

\section{実験設定}
\label{sec:experiment_setting}

\subsection{使用モデル}
Madryらが公開しているリポジトリ\cite{MadryMNIST, MadryCIFAR10}から入手した事前学習済みモデルを使用する.表\ref{table:models_detail}に各モデルの詳細を示す.

\begin{table}[hbtp]
  \caption{使用モデルの詳細}
  \label{table:models_detail}
  \centering
  \begin{tabular}{l|l|cc}
    \hline
    モデル名 & 訓練方法 & MNIST & CIFAR10 \\
    \hline
    nat & 自然学習 & \checkmark & \checkmark \\
    adv & 敵対的学習 & \checkmark & \checkmark \\
    nat\_and\_adv & 混合学習 & \checkmark & - \\
    weak\_adv & 弱敵対的学習 & \checkmark & - \\
    \hline
  \end{tabular}
\end{table}

\subsection{攻撃パラメータ}
表\ref{table:attack_params}にPGD攻撃のパラメータを示す.これらのパラメータはMadryらの論文\cite{PGD}と同一の設定である.

\begin{table}[hbtp]
  \caption{PGD攻撃のパラメータ設定}
  \label{table:attack_params}
  \centering
  \begin{tabular}{l|cc}
    \hline
    パラメータ & MNIST & CIFAR10 \\
    \hline
    摂動制約$\varepsilon$ & $0.3$ & $8/255 \approx 0.031$ \\
    ステップサイズ$\alpha$ & $0.01$ & $2/255 \approx 0.008$ \\
    反復数$T$ & $100$ & $100$ \\
    リスタート数 & $9$ & $9$ \\
    \hline
  \end{tabular}
\end{table}

\subsection{初期化手法}
本実験では,以下の3種類の初期化手法を比較する:
\begin{itemize}
    \item \textbf{Random}:$\ell_\infty$制約範囲内での一様乱数による初期化
    \item \textbf{DeepFool}:DeepFoolによる敵対的サンプルを初期点とする手法
    \item \textbf{Multi-DeepFool}:複数ターゲットへのDeepFoolの中から最大損失の点を選択する手法
\end{itemize}

DeepFoolのパラメータは,最大反復回数$T_{\text{df}} = 50$,オーバーシュート係数$\eta = 0.02$とする.

\subsection{実行環境}
% TODO: 実行環境の詳細を記載
実験はPython 3.xおよびPyTorchを用いて実施した.

%%%%%%%%%%%%%%%%%%%%%%%%%%%%%%%%%%%%%%%%%%%%%%%%%%%%%%%%%%%%%%%%%%%%%%%%%%%%%%%
\section{ランダム初期化PGDの収束解析}
\label{sec:random_init_analysis}

本節では,ランダム初期化PGD攻撃の収束挙動を分析する.これはMadryらの実験の再現及び定量的拡張に相当する.

\subsection{MNISTでの結果}
\label{sec:random_mnist}

図\ref{fig:mnist_random_loss_curves}に,MNISTの各モデルに対するランダム初期化PGD攻撃の損失曲線を示す.各図は1つのサンプルに対する複数リスタートの損失推移を表しており,下部のヒートマップは各リスタートにおける予測の正誤を示す.

% TODO: 図を挿入(figures/run_all/mnist_*_random_*.png)
\begin{figure}[htbp]
  \centering
  % \includegraphics[width=\linewidth]{figure/run_all/mnist_nat_random_sample.png}
  \caption{MNIST natモデルに対するランダム初期化PGDの損失曲線(図は実験完了後に挿入)}
  \label{fig:mnist_nat_random}
\end{figure}

\begin{figure}[htbp]
  \centering
  % \includegraphics[width=\linewidth]{figure/run_all/mnist_adv_random_sample.png}
  \caption{MNIST advモデルに対するランダム初期化PGDの損失曲線}
  \label{fig:mnist_adv_random}
\end{figure}

\begin{figure}[htbp]
  \centering
  % \includegraphics[width=\linewidth]{figure/run_all/mnist_nat_and_adv_random_sample.png}
  \caption{MNIST nat\_and\_advモデルに対するランダム初期化PGDの損失曲線}
  \label{fig:mnist_nat_and_adv_random}
\end{figure}

\begin{figure}[htbp]
  \centering
  % \includegraphics[width=\linewidth]{figure/run_all/mnist_weak_adv_random_sample.png}
  \caption{MNIST weak\_advモデルに対するランダム初期化PGDの損失曲線}
  \label{fig:mnist_weak_adv_random}
\end{figure}

表\ref{table:mnist_random_convergence}に,MNISTの各モデルにおける収束統計量を示す.

\begin{table}[hbtp]
  \caption{MNISTランダム初期化PGDの収束統計(閾値$\theta = 0.90$)}
  \label{table:mnist_random_convergence}
  \centering
  \begin{tabular}{l|cccc}
    \hline
    モデル & 収束率 & 平均 & 中央値 & P95 \\
    \hline
    nat & - & - & - & - \\
    adv & - & - & - & - \\
    nat\_and\_adv & - & - & - & - \\
    weak\_adv & - & - & - & - \\
    \hline
  \end{tabular}
\end{table}

\subsection{CIFAR10での結果}
\label{sec:random_cifar10}

図\ref{fig:cifar10_random_loss_curves}に,CIFAR10の各モデルに対するランダム初期化PGD攻撃の損失曲線を示す.

% TODO: 図を挿入(figures/run_all/cifar10_*_random_*.png)
\begin{figure}[htbp]
  \centering
  % \includegraphics[width=\linewidth]{figure/run_all/cifar10_nat_random_sample.png}
  \caption{CIFAR10 natモデルに対するランダム初期化PGDの損失曲線}
  \label{fig:cifar10_nat_random}
\end{figure}

\begin{figure}[htbp]
  \centering
  % \includegraphics[width=\linewidth]{figure/run_all/cifar10_adv_random_sample.png}
  \caption{CIFAR10 advモデルに対するランダム初期化PGDの損失曲線}
  \label{fig:cifar10_adv_random}
\end{figure}

表\ref{table:cifar10_random_convergence}に,CIFAR10の各モデルにおける収束統計量を示す.

\begin{table}[hbtp]
  \caption{CIFAR10ランダム初期化PGDの収束統計(閾値$\theta = 0.90$)}
  \label{table:cifar10_random_convergence}
  \centering
  \begin{tabular}{l|cccc}
    \hline
    モデル & 収束率 & 平均 & 中央値 & P95 \\
    \hline
    nat & - & - & - & - \\
    adv & - & - & - & - \\
    \hline
  \end{tabular}
\end{table}

\subsection{データセット・モデル間比較}
\label{sec:random_comparison}

図\ref{fig:random_mean_loss_overlay}に,各モデルの平均正規化損失曲線を示す.この図により,モデル間での収束速度の違いを比較できる.

% TODO: 図を挿入(convergence_analysis/threshold_0_90/*/random/mean_loss_overlay.png)
\begin{figure}[htbp]
  \centering
  % \includegraphics[width=0.48\linewidth]{figure/convergence/mnist_random_mean_loss.png}
  % \includegraphics[width=0.48\linewidth]{figure/convergence/cifar10_random_mean_loss.png}
  \caption{ランダム初期化PGDの平均正規化損失曲線(左:MNIST,右:CIFAR10)}
  \label{fig:random_mean_loss_overlay}
\end{figure}

図\ref{fig:random_convergence_cdf}に,収束反復数の累積分布関数(CDF)を示す.

% TODO: 図を挿入(convergence_analysis/threshold_0_90/*/random/convergence_cdf.png)
\begin{figure}[htbp]
  \centering
  % \includegraphics[width=0.48\linewidth]{figure/convergence/mnist_random_cdf.png}
  % \includegraphics[width=0.48\linewidth]{figure/convergence/cifar10_random_cdf.png}
  \caption{ランダム初期化PGDの収束反復数CDF(左:MNIST,右:CIFAR10)}
  \label{fig:random_convergence_cdf}
\end{figure}

%%%%%%%%%%%%%%%%%%%%%%%%%%%%%%%%%%%%%%%%%%%%%%%%%%%%%%%%%%%%%%%%%%%%%%%%%%%%%%%
\section{DeepFool初期化の収束解析}
\label{sec:deepfool_init_analysis}

本節では,DeepFool初期化PGD攻撃の収束挙動を分析し,ランダム初期化との比較を行う.

\subsection{MNISTでの結果}
\label{sec:deepfool_mnist}

図\ref{fig:mnist_deepfool_loss_curves}に,MNISTの各モデルに対するDeepFool初期化PGD攻撃の損失曲線を示す.

% TODO: 図を挿入(figures/run_all/mnist_*_deepfool_*.png)
\begin{figure}[htbp]
  \centering
  % \includegraphics[width=\linewidth]{figure/run_all/mnist_nat_deepfool_sample.png}
  \caption{MNIST natモデルに対するDeepFool初期化PGDの損失曲線}
  \label{fig:mnist_nat_deepfool}
\end{figure}

\begin{figure}[htbp]
  \centering
  % \includegraphics[width=\linewidth]{figure/run_all/mnist_adv_deepfool_sample.png}
  \caption{MNIST advモデルに対するDeepFool初期化PGDの損失曲線}
  \label{fig:mnist_adv_deepfool}
\end{figure}

表\ref{table:mnist_deepfool_convergence}に,MNISTの各モデルにおける収束統計量を示す.

\begin{table}[hbtp]
  \caption{MNIST DeepFool初期化PGDの収束統計(閾値$\theta = 0.90$)}
  \label{table:mnist_deepfool_convergence}
  \centering
  \begin{tabular}{l|cccc}
    \hline
    モデル & 収束率 & 平均 & 中央値 & P95 \\
    \hline
    nat & - & - & - & - \\
    adv & - & - & - & - \\
    nat\_and\_adv & - & - & - & - \\
    weak\_adv & - & - & - & - \\
    \hline
  \end{tabular}
\end{table}

\subsection{CIFAR10での結果}
\label{sec:deepfool_cifar10}

図\ref{fig:cifar10_deepfool_loss_curves}に,CIFAR10の各モデルに対するDeepFool初期化PGD攻撃の損失曲線を示す.

% TODO: 図を挿入(figures/run_all/cifar10_*_deepfool_*.png)
\begin{figure}[htbp]
  \centering
  % \includegraphics[width=\linewidth]{figure/run_all/cifar10_nat_deepfool_sample.png}
  \caption{CIFAR10 natモデルに対するDeepFool初期化PGDの損失曲線}
  \label{fig:cifar10_nat_deepfool}
\end{figure}

\begin{figure}[htbp]
  \centering
  % \includegraphics[width=\linewidth]{figure/run_all/cifar10_adv_deepfool_sample.png}
  \caption{CIFAR10 advモデルに対するDeepFool初期化PGDの損失曲線}
  \label{fig:cifar10_adv_deepfool}
\end{figure}

表\ref{table:cifar10_deepfool_convergence}に,CIFAR10の各モデルにおける収束統計量を示す.

\begin{table}[hbtp]
  \caption{CIFAR10 DeepFool初期化PGDの収束統計(閾値$\theta = 0.90$)}
  \label{table:cifar10_deepfool_convergence}
  \centering
  \begin{tabular}{l|cccc}
    \hline
    モデル & 収束率 & 平均 & 中央値 & P95 \\
    \hline
    nat & - & - & - & - \\
    adv & - & - & - & - \\
    \hline
  \end{tabular}
\end{table}

\subsection{PGD制約$\varepsilon$変化による影響}
\label{sec:eps_variation}

DeepFoolによる摂動は$\ell_2$ノルムで最小化されるため,PGD攻撃の$\ell_\infty$制約$\varepsilon$との関係が収束に影響を与える可能性がある.本節では,$\varepsilon$を変化させた場合の影響を調査する.

表\ref{table:eps_variation}に,$\varepsilon$を標準設定の1倍,2倍,3倍とした場合の収束統計を示す.

\begin{table}[hbtp]
  \caption{PGD制約$\varepsilon$変化による収束統計の変化(MNIST nat)}
  \label{table:eps_variation}
  \centering
  \begin{tabular}{l|c|cccc}
    \hline
    $\varepsilon$倍率 & $\varepsilon$値 & 収束率 & 平均 & 中央値 & P95 \\
    \hline
    1倍 & 0.3 & - & - & - & - \\
    2倍 & 0.6 & - & - & - & - \\
    3倍 & 0.9 & - & - & - & - \\
    \hline
  \end{tabular}
\end{table}

% TODO: 図を挿入(figures/run_df_eps/mnist_nat_*_eps*.png)

%%%%%%%%%%%%%%%%%%%%%%%%%%%%%%%%%%%%%%%%%%%%%%%%%%%%%%%%%%%%%%%%%%%%%%%%%%%%%%%
\section{Multi-DeepFool初期化の収束解析}
\label{sec:multi_deepfool_analysis}

本節では,Multi-DeepFool初期化PGD攻撃の収束挙動を分析する.

\subsection{MNISTでの結果}
\label{sec:mdf_mnist}

図\ref{fig:mnist_mdf_loss_curves}に,MNISTの各モデルに対するMulti-DeepFool初期化PGD攻撃の損失曲線を示す.

% TODO: 図を挿入(figures/run_all/mnist_*_multi_deepfool_*.png)
\begin{figure}[htbp]
  \centering
  % \includegraphics[width=\linewidth]{figure/run_all/mnist_nat_multi_deepfool_sample.png}
  \caption{MNIST natモデルに対するMulti-DeepFool初期化PGDの損失曲線}
  \label{fig:mnist_nat_mdf}
\end{figure}

表\ref{table:mnist_mdf_convergence}に,MNISTの各モデルにおける収束統計量を示す.

\begin{table}[hbtp]
  \caption{MNIST Multi-DeepFool初期化PGDの収束統計(閾値$\theta = 0.90$)}
  \label{table:mnist_mdf_convergence}
  \centering
  \begin{tabular}{l|cccc}
    \hline
    モデル & 収束率 & 平均 & 中央値 & P95 \\
    \hline
    nat & - & - & - & - \\
    adv & - & - & - & - \\
    nat\_and\_adv & - & - & - & - \\
    weak\_adv & - & - & - & - \\
    \hline
  \end{tabular}
\end{table}

\subsection{CIFAR10での結果}
\label{sec:mdf_cifar10}

図\ref{fig:cifar10_mdf_loss_curves}に,CIFAR10の各モデルに対するMulti-DeepFool初期化PGD攻撃の損失曲線を示す.

% TODO: 図を挿入(figures/run_all/cifar10_*_multi_deepfool_*.png)
\begin{figure}[htbp]
  \centering
  % \includegraphics[width=\linewidth]{figure/run_all/cifar10_nat_multi_deepfool_sample.png}
  \caption{CIFAR10 natモデルに対するMulti-DeepFool初期化PGDの損失曲線}
  \label{fig:cifar10_nat_mdf}
\end{figure}

表\ref{table:cifar10_mdf_convergence}に,CIFAR10の各モデルにおける収束統計量を示す.

\begin{table}[hbtp]
  \caption{CIFAR10 Multi-DeepFool初期化PGDの収束統計(閾値$\theta = 0.90$)}
  \label{table:cifar10_mdf_convergence}
  \centering
  \begin{tabular}{l|cccc}
    \hline
    モデル & 収束率 & 平均 & 中央値 & P95 \\
    \hline
    nat & - & - & - & - \\
    adv & - & - & - & - \\
    \hline
  \end{tabular}
\end{table}

\subsection{DeepFool反復回数の影響}
\label{sec:dfiter_variation}

Multi-DeepFool初期化において,DeepFoolの反復回数$T_{\text{df}}$が収束に与える影響を調査する.

表\ref{table:dfiter_variation}に,$T_{\text{df}}$を変化させた場合の収束統計を示す.

\begin{table}[hbtp]
  \caption{DeepFool反復回数変化による収束統計の変化(MNIST nat)}
  \label{table:dfiter_variation}
  \centering
  \begin{tabular}{c|cccc}
    \hline
    $T_{\text{df}}$ & 収束率 & 平均 & 中央値 & P95 \\
    \hline
    1 & - & - & - & - \\
    10 & - & - & - & - \\
    30 & - & - & - & - \\
    50 & - & - & - & - \\
    \hline
  \end{tabular}
\end{table}

% TODO: 図を挿入(figures/run_mdf_iter/mnist_nat_multi_deepfool_*_dfiter*_*.png)

%%%%%%%%%%%%%%%%%%%%%%%%%%%%%%%%%%%%%%%%%%%%%%%%%%%%%%%%%%%%%%%%%%%%%%%%%%%%%%%
\section{初期化手法間の比較}
\label{sec:init_comparison}

本節では,3種類の初期化手法(Random, DeepFool, Multi-DeepFool)の収束性能を比較する.

図\ref{fig:init_comparison_cdf}に,各初期化手法の収束反復数CDFを示す.

% TODO: 図を挿入(全初期化手法のCDFを重ねたグラフ)
\begin{figure}[htbp]
  \centering
  % \includegraphics[width=0.48\linewidth]{figure/convergence/mnist_init_comparison_cdf.png}
  % \includegraphics[width=0.48\linewidth]{figure/convergence/cifar10_init_comparison_cdf.png}
  \caption{初期化手法別の収束反復数CDF比較(左:MNIST,右:CIFAR10)}
  \label{fig:init_comparison_cdf}
\end{figure}

表\ref{table:init_comparison_mnist}および表\ref{table:init_comparison_cifar10}に,初期化手法別の収束統計を示す.

\begin{table}[hbtp]
  \caption{MNIST 初期化手法別比較(閾値$\theta = 0.90$)}
  \label{table:init_comparison_mnist}
  \centering
  \begin{tabular}{l|l|cccc}
    \hline
    モデル & 初期化手法 & 収束率 & 平均 & 中央値 & P95 \\
    \hline
    \multirow{3}{*}{nat} & Random & - & - & - & - \\
    & DeepFool & - & - & - & - \\
    & Multi-DeepFool & - & - & - & - \\
    \hline
    \multirow{3}{*}{adv} & Random & - & - & - & - \\
    & DeepFool & - & - & - & - \\
    & Multi-DeepFool & - & - & - & - \\
    \hline
  \end{tabular}
\end{table}

\begin{table}[hbtp]
  \caption{CIFAR10 初期化手法別比較(閾値$\theta = 0.90$)}
  \label{table:init_comparison_cifar10}
  \centering
  \begin{tabular}{l|l|cccc}
    \hline
    モデル & 初期化手法 & 収束率 & 平均 & 中央値 & P95 \\
    \hline
    \multirow{3}{*}{nat} & Random & - & - & - & - \\
    & DeepFool & - & - & - & - \\
    & Multi-DeepFool & - & - & - & - \\
    \hline
    \multirow{3}{*}{adv} & Random & - & - & - & - \\
    & DeepFool & - & - & - & - \\
    & Multi-DeepFool & - & - & - & - \\
    \hline
  \end{tabular}
\end{table}

%%%%%%%%%%%%%%%%%%%%%%%%%%%%%%%%%%%%%%%%%%%%%%%%%%%%%%%%%%%%%%%%%%%%%%%%%%%%%%%
\section{総合考察}
\label{sec:discussion}

本節では,実験結果を総合的に考察する.

\subsection{Madryらの実験の再現性}
ランダム初期化PGD攻撃の損失曲線は,Madryらの報告と同様に,異なる初期点から開始しても同程度のプラトーに収束する傾向が確認された.

% TODO: 実験結果に基づいて詳細な考察を追記

\subsection{収束速度の定量的評価}
90\%収束閾値を用いた評価では,以下の傾向が観察された:
\begin{itemize}
    \item 自然学習モデル(nat)は敵対的学習モデル(adv)と比較して収束が早い傾向がある
    \item MNISTとCIFAR10で同様の傾向が確認された
\end{itemize}

% TODO: 実験結果に基づいて具体的な数値を追記

\subsection{DeepFool初期化の効果}
DeepFool初期化の効果について,以下の知見が得られた:
\begin{itemize}
    \item 自然学習モデルでは,DeepFool初期化により収束が早まる傾向が見られた
    \item 敵対的学習モデルでは,効果が限定的であった
\end{itemize}

% TODO: 実験結果に基づいて詳細な考察を追記

\subsection{計算コストの観点からの評価}
DeepFool初期化による収束の高速化と,DeepFool自体の計算コストのトレードオフについて考察する.

% TODO: 実験結果に基づいて計算コストの考察を追記

