\documentclass[uplatex, 11pt]{jsarticle}
\usepackage{yokou}

\type{卒業論文}
\title{画像分類モデルへのPGD攻撃における\\初期化手法の検証}
\author{執行 凱斗}
\superviser{竹内 純一 教授}
\date{令和8年2月16日(月)13:15 $\sim$ 13:30}
\place{ウエスト2号館第3会議室(W2-725)}

\newlength{\secsep}
\setlength{\secsep}{1zh}

\begin{document}

\maketitle

\textbf{研究目的:}
Madryらが提案したPGD(Projected Gradient Descent)攻撃の検証を誤分類の観点から拡張し,攻撃対象モデルや初期化手法による誤分類特性の違いを評価する.また,別の攻撃手法であるDeepFoolを用いた新たな初期化手法を提案し,その効果を検証する.

\vspace{\secsep}
\textbf{技術的背景:}
画像分類モデルは,入力画像に微小なノイズ(敵対的摂動)を加えることで誤分類させられる場合がある.Madryらは,画像に対する損失を制約付きで最大化する摂動を反復法で求める手法であるPGD攻撃を提案し,ランダムな初期点から開始(ランダム初期化)しても損失が同程度の局所最大解に収束することを確認した.しかし,実際に誤分類するかどうかや,異なるモデルや初期化手法での誤分類に必要な反復数の違いについては触れられていない.

\vspace{\secsep}
\textbf{検討した課題:}
本研究では,以下の課題に取り組んだ.\\
- Madryらの実験を再現した上で,誤分類達成反復数(誤分類に達した反復数)に着目し,損失の推移だ
  けでなく誤分類の達成過程も評価した.\\
- MNIST,CIFAR10の2つのデータセットに対し,訓練に用いる敵対的サンプルの割合や強度が異なるモデルと,クリーン初期化(初期点が入力画像)・ランダム初期化・DeepFoolを用いた初期化(提案手法)の全組み合わせで実験を行い,誤分類達成反復数を比較した.

\vspace{\secsep}
\textbf{結果:}\\
- Madry らの再現実験において,MNIST,CIFAR10 ともに損失曲線は同じように収束した一方で,CIFAR10 は MNIST よりはるかに少ない反復数で誤分類を達成した.\\
- DeepFoolを用いた初期化は,十分に堅牢なモデルに対しては効果が限定的であったが,それ以外のモデルに対してはランダム初期化より少ない反復数で誤分類した.これにより,初期化手法の工夫で敵対的攻撃の計算時間を減らせる可能性を示した.

\end{document}
